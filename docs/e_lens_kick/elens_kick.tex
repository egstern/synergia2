\documentclass[acus]{jacow}

\usepackage[ascii]{inputenc}
\usepackage{graphicx}

%\pagestyle{empty}

\begin{document}

\title{Momentum Kick to a Proton Traversing an Electron Lens}

\author{Eric G. Stern}

\maketitle

\begin{abstract}
  I calculate the momentum kick for a proton passing through an electron lens
  with a cylindrically round Gaussian or uniform density profile.
  This calculation is valid for non-relativistic protons.
\end{abstract}

\section{Introduction}

\begin{figure}[!htb]
   \centering
   \includegraphics*[width=\columnwidth]{e_lens_sketch.pdf}
   \caption{Diagram of electron lens configuration }
   \label{e_lens_sketch}
\end{figure}


I consider the momentum kick of a proton passing through an electron lens
of length L with a cylindrically round Gaussian density profile.
The proton flow direction is opposite to the electron progagation so the
EM force is focussing for protons to counteract the defocussing space charge
force.
The configuration is illustrated in~Fig.~\ref{e_lens_sketch}.

\begin{table}[!hbt]
   \centering
   \caption{Parameters and symbols used in the calculation}
   \begin{tabular}{lr}
       \toprule
       \textbf{Parameter} & \textbf{Definition} \\
       \midrule
       $J$       & Electron current \\
       $L$       & Electron lens length  \\
       $e$       & unit of electric charge \\
       $c$       & speed of light \\
       $m$       & proton mass \\
       $r_p$     & proton classical radius $e^2/4 \pi \epsilon_0 m c^2$ \\
       $\sigma$ & RMS radius of current distribution     \\
       $a$      & radius of the uniform current distribution \\
       $\beta_e$    & electron velocity/$c$   \\
       $\beta_p$  & proton velocity/$c$ \\
       $\beta_b$  & reference proton beam velocity/$c$ \\
       $\gamma_b$ & reference proton beam relativistic factor \\
       \bottomrule
   \end{tabular}
   \label{parameters}
\end{table}
  

The names and symbols of parameters used in the calculation are shown in
Table~\ref{parameters}.

\section{Gaussian electron beam}

The electric and magnetic fields experienced by a particle at radius~$r$ are
determined the amount of charge and current countained within the cylinder
of radius~$r$.
For the Gaussian profile, this is:
$$
\int_0^{r} e^{-{{r^2} / { 2 \sigma^2}}} r dr = 1 - e^{-{{r^2} / {2 \sigma^2}}}
$$

Using 2D Gauss's law, the normal electric field in the electron lens at radius~$r$ is
$$
E_n = -{ J \over {2 \pi \epsilon_0 \beta_e c}} ( 1 - e^{-{{r^2} / {2 \sigma^2}}} ) {1 \over r}
$$
identifying the charge density is $J / \beta_e c$.
The electric momentum kick $q e E \Delta t$ for a single charge over length $L$ is
$$
\Delta p_E = -{ J e \over {2 \pi \epsilon_0 \beta_e c}} ( 1 - e^{-{{r^2} / {2 \sigma^2}}} )  {1 \over r}{ L \over { \beta_p c }}
$$

The magnetic field from the 2D Ampere's law is
$$
B_\theta = {  J \over {2 \pi \epsilon_0 c^2}} ( 1 - e^{-{{r^2} / {2 \sigma^2}}} )  {1 \over r}
$$
where I have used the identity $\epsilon_0 \mu_0 = 1/c^2$.
The magnetic momentum kick $q v \times B \Delta t$ for a single charge over length $L$ is
$$
\Delta p_M = - {  J e \beta_p c \over {2 \pi \epsilon_0 c^2}} ( 1 - e^{-{{r^2} / {2 \sigma^2}}} )  {1 \over r}{ L \over { \beta_p c }}
$$

The total momentum kick is the sum of the electric and magnetic kicks:
$$
\Delta p_{E+M} = - {  J L e (1 + \beta_e \beta_p) \over {2 \pi \epsilon_0 \beta_e \beta_p c^2}} ( 1 - e^{-{{r^2} / {2 \sigma^2}}} ) {1 \over r}
$$

In Synergia, the particle momentum is normalized by the reference beam momentum, given by $m \beta_b \gamma_b c$.
The momentum kick can be expressed by
$$
{\Delta p \over p_b} = - {  J L e (1 + \beta_e \beta_p) \over {2 \pi \epsilon_0 m c^2 \beta_e \beta_p \beta_b \gamma_b c}} ( 1 - e^{-{{r^2} / {2 \sigma^2}}} ) {1 \over r}
$$
which can be rewritten using the proton classical radius $r_p$ as
$$
{\Delta p \over p_b} = - {  2 J L r_p (1 + \beta_e \beta_p) \over {e \beta_e \beta_p \beta_b \gamma_b c}} ( 1 - e^{-{{r^2} / {2 \sigma^2}}} ) {1 \over r}
$$

\section{Uniform electron beam}
In the case of a uniform electron beam, the factor $( 1 - e^{-{{r^2} / {2 \sigma^2}}} )$ is replaced by
$r^2/a^2$.
The kick becomes
$$
{\Delta p \over p_b} = - {  2 J L r_p (1 + \beta_e \beta_p) \over {e \beta_e \beta_p \beta_b \gamma_b c}} {r \over a^2}
$$
In Eq.~2 of Ref.\ref{shiltsev}, the kick on an ultra-relativistic proton from a
uniform distribution electron lens is described as being caused by a
potential function $V(r)$ of the form
$$
V(r) = r^2 {{(1 + \beta_e ) J L r_p } \over {e \beta_e c a^2 \gamma_p}}
$$
Calculating the kick as $ -\partial V / \partial r $ , this matches the  expression calculated
in this paper after
setting $\beta_p$ to 1.

\begin{thebibliography}{9}   % Use for  1-9  references
\bibitem{shiltsev}
  Shiltsev, et. al, ``Considerations on Compensation of Beam-Beam Effects,''
  Phys. Rev. ST Accel Beams, \bf{2}, 171001 (1999).
\end{thebibliography}
\end{document}
